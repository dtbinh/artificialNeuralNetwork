\chapter{class Neuron}
Uses step function for calculating the output.

\section{Attributes}
\subsection{float[] inputs}
For each input one float variable is used, which stores the value given by the \texttt{setInput()}-Method.

\subsection{float netInput}
The result of the net input function, respectively the sum of all inputs.

\subsection{float threshold}
The value when the output will be 0.5.

\subsection{float output}
So far, since the output function is the identity, the variable contains the result of the the activation function.

\section{Constructor}
\subsection{Neuron(int nrInputs)}
Creates \texttt{inputs} by the given value \texttt{nrInputs} and initialzes \texttt{inputs}, \texttt{netInput}, \texttt{output} to 0, and \texttt{threshold} to max ($0x7fffffff$).

\section{Methods}
\subsection{Boolean setInput(int nrInput, float value)}
Sets given input to the given value.\\
Returns true if input was set, respectively if \texttt{nrInput} is smaller than the length of the \texttt{inputs}-array.
	
\subsection{int getNumberOfInputs()}
Returns the length of the \texttt{inputs}-array.

\subsection{float getThreshold()}
Returns the value of \texttt{threshold}.

\subsection{void setThreshold(float threshold)}
Sets \texttt{threshold} to the given value.

\subsection{float getOutput()}
Calculates and returns the \texttt{output}. First \texttt{netInput} is calculated by summing up all inputs. Since the output function is just the identity the \texttt{output} is calculated by using the activation function, which is a logistic function: $output = 1 / (1 + e^{(-(netInput - threshold))})$.

\chapter{class Connection}
Connects the output of a neuron to an input of another neuron. The weight of the following input is the output value times the weight of the connection.

\section{Attributes}
\subsection{Neuron neuronFrom}
Neuron the output is taken from.

\subsection{Neuron neuronTo}
Neuron the input is set.

\subsection{int input}
Input of \texttt{neuronTo} which is set.

\subsection{float connWeight}
Weight of the connection.

\subsection{float weightToSet}
Weight set in \texttt{neuronTo}. Output of \texttt{neuronFrom} $*$ weight of connection.

\section{Constructors}
\subsection{Connection(Neuron neuronFrom, Neuron neuronTo, int input, float connWeight)}
Just sets \texttt{neuronFrom}, \texttt{neuronTo}, \texttt{input}, and \texttt{connWeight}.

\section{Methods}
\subsection{void run()}
Calculates the weight for \texttt{neuronTo}. The weight, which is set, is calculated by \texttt{getOutput()} from \texttt{neuronFrom} $*$ the weight of the connection.for \texttt{neuronTo}. The calculated weight is set to the given input.\\
The method uses \texttt{getActivation()} from \texttt{neuronFrom} to set or unset the given input of \texttt{neuronTo}.